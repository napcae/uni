 %%%%%%%%%%%%%%%%%%%%%%%%%%%%%%%%%%%%%%%%%%%%%%%%%%%%%%%%%%%%
%%  This Beamer template was created by Cameron Bracken.
%%  Anyone can freely use or modify it for any purpose
%%  without attribution.
%%
%%  Last Modified: January 9, 2009
%%
%%  http://cameron.bracken.bz/beamer-template

\documentclass[xcolor=x11names,compress]{beamer}
\usepackage{pgfpages}

%% General document %%%%%%%%%%%%%%%%%%%%%%%%%%%%%%%%%%
\usepackage{graphicx}
\usepackage{tikz}
\usepackage{german}
\usetikzlibrary{decorations.fractals}
%%%%%%%%%%%%%%%%%%%%%%%%%%%%%%%%%%%%%%%%%%%%%%%%%%%%%
%\setbeameroption{show notes on second screen=left}
%\setbeamertemplate{note page}{%
%  \insertnote%
%}
%% Beamer Layout %%%%%%%%%%%%%%%%%%%%%%%%%%%%%%%%%%
\useoutertheme[subsection=false,shadow]{miniframes}
\useinnertheme{default}
\setbeamertemplate{footline}{%
\begin{beamercolorbox}{section in head/foot}
    \color{gray}\vskip2pt~  \insertshorttitle\hfill\insertpagenumber{} %
    of \insertpresentationendpage{} ~\vskip2pt
\end{beamercolorbox}
}
\usefonttheme{serif}
\usepackage{palatino}

\setbeamerfont{title like}{shape=\scshape}
\setbeamerfont{frametitle}{shape=\scshape}

\setbeamercolor*{lower separation line head}{bg=DeepSkyBlue4} 
\setbeamercolor*{normal text}{fg=black,bg=white} 
\setbeamercolor*{alerted text}{fg=red} 
\setbeamercolor*{example text}{fg=black} 
\setbeamercolor*{structure}{fg=black} 
 
\setbeamercolor*{palette tertiary}{fg=black,bg=black!10} 
\setbeamercolor*{palette quaternary}{fg=black,bg=black!10} 

\setbeamertemplate{note page}[plain]


\renewcommand{\(}{\begin{columns}}
\renewcommand{\)}{\end{columns}}
\newcommand{\<}[1]{\begin{column}{#1}}
\renewcommand{\>}{\end{column}}
%%%%%%%%%%%%%%%%%%%%%%%%%%%%%%%%%%%%%%%%%%%%%%%%%%
\title[Secure Hash Algorithm]{Secure Hash Algorithm}



\begin{document}

%%%%%%%%%%%%%%%%%%%%%%%%%%%%%%%%%%%%%%%%%%%%%%%%%%%%%%
%%%%%%%%%%%%%%%%%%%%%%%%%%%%%%%%%%%%%%%%%%%%%%%%%%%%%%
\begin{frame}
\title{Secure Hash Algorithm}
%\subtitle{SUBTITLE}
\subtitle{SHA-256}
\author{
	Chi Trung Nguyen\\
	{\it T-Systems}\\
}
\date{
	\begin{tikzpicture}[decoration=Koch curve type 2] 
		\draw[DeepSkyBlue4] decorate{ decorate{ decorate{ (0,0) -- (3,0) }}}; 
	\end{tikzpicture}  
	\\
	\vspace{1cm}
	\today
}
\titlepage
\end{frame}

%%%%%%%%%%%%%%%%%%%%%%%%%%%%%%%%%%%%%%%%%%%%%%%%%%%%%%
%%%%%%%%%%%%%%%%%%%%%%%%%%%%%%%%%%%%%%%%%%%%%%%%%%%%%%
\begin{frame}{Agenda}
%\tiny
\tableofcontents[pausesections]
\note[item]{um erinnerung etwas aufzufrischen: Was ist ein hash}
\note[item]{geschichte gegliedert in einzelne mitglieder}
\note[item]{implementierung sha256 und anwendungen von hashalgorithmen bzw sha2}
\note[item]{schlussendlich kleiner ausblick auf sha3 der im sommer 2012 vorgestellt werden soll}
\end{frame}

%%%%%%%%%%%%%%%%%%%%%%%%%%%%%%%%%%%%%%%%%%%%%%%%%%%%%%
%%%%%%%%%%%%%%%%%%%%%%%%%%%%%%%%%%%%%%%%%%%%%%%%%%%%%%

\section{\scshape Einf"uhrung}
\subsection{Was ist ein Hash?}
\begin{frame}{Was ist ein Hash?}

\begin{itemize}
\item deutsch:  \glqq\textit{zerhacken}\grqq, \glqq \textit{verstreuen}\grqq
	\pause
\item Hashfunktion oder Streuwertfunktion erstellt aus beliebiger gro"ser Quellmenge eine immer gleich gro"se Zielmenge
\note[item]{bsp automarken, mercedes = 1,bmw = 2}
\begin{itemize}
\item $ f(x) = f(x') $
	\pause
\end{itemize}
\item Einwegfunktion

\note[item]{krypto hash, hash unterschied: hash beliebig abgebildet(zb Perl), kryptohash = berechnet mit algorithmus}
\end{itemize}
\end{frame}

%%%%%%%%%%%%%%%%%%%%%%%%%%%%%%%%%%%%%%%%%%%%%%%%%%%%%%
%%%%%%%%%%%%%%%%%%%%%%%%%%%%%%%%%%%%%%%%%%%%%%%%%%%%%%

\section{\scshape Geschichte}
\subsection{SHA Allgemein}
\begin{frame}{SHA Allgemein}
%The Secure Hash Algorithm is one of a number of cryptographic hash %functions published by the National Institute of Standards and Technology %(NIST) as a U.S. Federal Information Processing Standard (FIPS)
\begin{itemize}


\note[item]{sha-0 wurde...}
\item 1993 vom {\bf National Institute of Standards (NIST)} 
als ein {\bf U.S. Federal Information Processing Standard (FIPS)} 
ver"offentlicht

	\pause
\note[item]{sha steht f"ur secure hash algorithm und ist eine gruppe von hash algorithmn}
\item Gruppe von kryptologischer Hashfunktionen
	\begin{itemize}
		\item SHA-0	
		\item SHA-1
		\item SHA-2
		\item SHA-3
	\end{itemize}

\end{itemize}
\end{frame}

%%%%%%%%%%%%%%%%%%%%%%%%%%%%%%%%%%%%%%%%%%%%%%%%%%%%%%
%%%%%%%%%%%%%%%%%%%%%%%%%%%%%%%%%%%%%%%%%%%%%%%%%%%%%%
\subsection{SHA-0}
\begin{frame}{SHA-0}
\begin{itemize}
\item 1993 ver"offentlicht
	\pause
\note[item]{urspr"unglich als}
\item Bestandteil des Digital Signature Algorithms (DSA) f"ur Digital Signature Standard (DSS)
\note[item]{1991 empfohlen}

\end{itemize}
\end{frame}

%%%%%%%%%%%%%%%%%%%%%%%%%%%%%%%%%%%%%%%%%%%%%%%%%%%%%%
%%%%%%%%%%%%%%%%%%%%%%%%%%%%%%%%%%%%%%%%%%%%%%%%%%%%%%
\subsection{SHA-1}
\begin{frame}{SHA-1}
\begin{itemize}
\item 1995 ver"offentlicht
\pause
\item aufgrund Designfehler in SHA-0
\note[item]{alles optional, nicht vorlesen!}
\note[item]{2005 von Xiaoyun Wang, Yiqun Lisa Yin und Hongbo Yu an der Shandong University in China gebrochen}
\note[item]{Ihnen war es gelungen, den Aufwand zur Kollisionsberechnung von $2^{80}$ auf $2^{69}$ zu verringern}
\note[item]{es wurde ein rechtsshift durch ein linksshift ersetzt}
\note[item]{August 2005, wurde von Xiaoyun Wang, Andrew Yao und Frances Yao auf der Konferenz CRYPTO 2005 ein weiterer, effizienterer Kollisionsangriff auf SHA-1 vorgestellt, welcher den Berechnungsaufwand auf $2^{63}$ reduziert}

\end{itemize}
\end{frame}


%%%%%%%%%%%%%%%%%%%%%%%%%%%%%%%%%%%%%%%%%%%%%%%%%%%%%%
%%%%%%%%%%%%%%%%%%%%%%%%%%%%%%%%%%%%%%%%%%%%%%%%%%%%%%
\subsection{SHA-2}
\begin{frame}{SHA-2}
\begin{itemize}
\item 2002 ver"offentlicht
\pause
\item existiert in mehreren Bit Variante
\note[item]{die hier gelistet sind(n"achste folie)}

%%TODO
%%MERKLE DEMGARD
%%EVENTUELL UNTERSCHIEDE ZU SHA0 und SHA1


\end{itemize}
\end{frame}

%%%%%%%%%%%%%%%%%%%%%%%%%%%%%%%%%%%%%%%%%%%%%%%%%%%%%%
%%%%%%%%%%%%%%%%%%%%%%%%%%%%%%%%%%%%%%%%%%%%%%%%%%%%%%
\begin{frame}[shrink=30]{}
\note[item]{$2^{64}$ bits = 2 147 483 648 gigabytes}
\note[item]{auch wenn $2^{128}$ bits an daten m"oglich sind, werden $2^{64}$ in der realit"at nicht "uberschritten}
\note[item]{unterschiede erkl"aren: andere konstanten bei 224 \& 384, worte werden weggelassen}
\begin{table}[c]
\caption{Secure Hash Algorithmus Eigenschaften}
\begin{tabular}[ht]{|c|p{0.2\textwidth}|p{0.2\textwidth}|p{0.2\textwidth}|p{0.2\textwidth}|}
  \hline
  Algorithmus &  Message Gr"o"se(bits) & Block Gr"o"se(bits) & Word Gr"o"se(bits) & Message Digest Gr"o"se(bits)\\
  \hline\hline
  SHA-1   & $<2^{64}$  &  512 & 32 & 160\\
  SHA-224 & $<2^{64}$  &  512 & 32 & 224\\
  SHA-256 & $<2^{64}$  &  512 & 32 & 256\\
  SHA-384 & $<2^{128}$ & 1024 & 64 & 384\\  
  SHA-512 & $<2^{128}$ & 1024 & 64 & 512\\  
  \hline
\end{tabular}
\label{tab:meinetabelle}
\end{table}
\end{frame}


%%%%%%%%%%%%%%%%%%%%%%%%%%%%%%%%%%%%%%%%%%%%%%%%%%%%%%
%%%%%%%%%%%%%%%%%%%%%%%%%%%%%%%%%%%%%%%%%%%%%%%%%%%%%%
\section{\scshape Implementierung}
\begin{frame}{Darstellung des Algorithmus}
\note[item]{SHA2 teilt erst in messages,dann in 64 byte chunks, wendet mathematische funktionen auf jeden chunk, und f"ugt diese dann dem hash hinzu.}
\note[item]{der algorithmus tut dies f"ur jeden chunk und f"ugt es dann dem hash wert hinzu.}
\note[item]{am ende bekommt man einen hash wert mit einer genauen l"ange}
\includegraphics[scale=0.3]{sha256.png}\\
\end{frame}

%%%%%%%%%%%%%%%%%%%%%%%%%%%%%%%%%%%%%%%%%%%%%%%%%%%%%%
%%%%%%%%%%%%%%%%%%%%%%%%%%%%%%%%%%%%%%%%%%%%%%%%%%%%%%
\subsection{Algorithmus}
\begin{frame}{Funktionen}
$ Ch(E,F,G) = (E\wedge F) \oplus (\neg E\wedge G)$
$ Maj(A,B,C) = (A\wedge B) \oplus (A\wedge C) \oplus (B\wedge C)$\\
$ \Sigma_0 = (A\ggg 2) \oplus (A\ggg 13) \oplus (A\ggg 22) $\\
$ \Sigma_1 = (A\ggg 6) \oplus (A\ggg 11) \oplus (A\ggg 25) $\\
\note[item]{rotate!!}
\end{frame}
%%%%%%%%%%%%%%%%%%%%%%%%%%%%%%%%%%%%%%%%%%%%%%%%%%%%%%
%%%%%%%%%%%%%%%%%%%%%%%%%%%%%%%%%%%%%%%%%%%%%%%%%%%%%%
\subsection{Pseudocode}
\begin{frame}{Pseudocode}
\begin{itemize}
\item Initialisiere Variabeln
(die ersten 32 Bits der Nachkommastellen der Quadratwurzeln von den ersten 8 Primzahlen 2..19):\\
\texttt{h[0..7] := 0x6a09e667,[...],0x5be0cd19}
\pause
\item Initialisiere Variabeln der Runden Konstanten
(die ersten 32 Bits der Nachkommastellen der Kubikwurzel von den ersten 64 Primzahlen 2..311):\\
\texttt{k[0..63] := 0x428a2f98,[...], 0xc67178f2}
\end{itemize}



\end{frame}
%%%%%%%%%%%%%%%%%%%%%%%%%%%%%%%%%%%%%%%%%%%%%%%%%%%%%%
%%%%%%%%%%%%%%%%%%%%%%%%%%%%%%%%%%%%%%%%%%%%%%%%%%%%%%

\begin{frame}{Preprocessing}
\begin{itemize}
\item \texttt{bit $1$ zur $message$ hinzuf"ugen} \pause
\item \texttt{anzahl von $k$ bits $0$ hinzuf"ugen, wobei $k$ die kleinst m"ogliche Zahl >= 0, so dass die L"ange der $message$ (in bits) Modulo 512 minus 64 bits ist} \pause
\item \texttt{L"ange der $message$(vor dem Preprocessing), in bits, als 64-bit big-endian integer hinzufügen}\newline \newline \pause
\item \texttt{$message$ in 512-bit chunks teilen}
\item \texttt{foreach chunk\{}\newline
    \texttt{teile chunk in sechzehn 32-bit big-endian Worte $w[0..15]$}
\note[item]{bsp nachricht \glqq abc \grqq}
\note[item]{representation in ascii: 97 = a, 98=b,99=c}
\note[item]{a= 0110 0001}
\note[item]{b= 0110 0010}
\note[item]{c= 0110 0011}
\note[item]{da 8x3 = 24 ergibt sich $448-(24+1)=423$ zero Bits}
\note[item]{am ende nachricht hinzufugen}
\note[item]{\includegraphics[scale=0.52]{padding.png}}
\end{itemize}




\end{frame}
%%%%%%%%%%%%%%%%%%%%%%%%%%%%%%%%%%%%%%%%%%%%%%%%%%%%%%
%%%%%%%%%%%%%%%%%%%%%%%%%%%%%%%%%%%%%%%%%%%%%%%%%%%%%%

\begin{frame}{Erweiterung der Worte}

\begin{itemize}

\item[]
\texttt{for $i=16$ to $63$ \{}
		 \begin{itemize}
		 \item[] $s0$ := ($w[i-15]$ rightrotate $7$) xor ($w[i-15]$ rightrotate $18$) xor ($w[i-15]$ rightshift $3$) \newline
        \item[] $s1$ := ($w[i-2]$ rightrotate $17$) xor ($w[i-2]$ rightrotate $19$) xor ($w[i-2]$ rightshift $10$) \newline
        \item[] $w[i]$ := $w[i-16]$ + $s0$ + $w[i-7]$ + $s1$ \newline \}
        \note{was ist ein rechtsrotate?was ist padding?}
		 \end{itemize}
		 %%aufschreiben was ist rechtsrotate,was ist rechtsshift + was ist das padding(?) erklären können!
         
\end{itemize}


\end{frame}
%%%%%%%%%%%%%%%%%%%%%%%%%%%%%%%%%%%%%%%%%%%%%%%%%%%%%%
%%%%%%%%%%%%%%%%%%%%%%%%%%%%%%%%%%%%%%%%%%%%%%%%%%%%%%

\begin{frame}{Hashzuweisung}
\texttt{$a := h0$ \\
    $b := h1$ \\
    $c := h2$ \\
    $d := h3$ \\
    $e := h4$ \\
    $f := h5$ \\
    $g := h6$ \\
    $h := h7$} 
%%h0 = nachkommastelle der quadratwurzel
\end{frame}
%%%%%%%%%%%%%%%%%%%%%%%%%%%%%%%%%%%%%%%%%%%%%%%%%%%%%%
%%%%%%%%%%%%%%%%%%%%%%%%%%%%%%%%%%%%%%%%%%%%%%%%%%%%%%

\begin{frame}{Hauptschleife}
\begin{itemize}[]
\item[]
\texttt{for $i=0$ to $63$ \{} \\
\begin{itemize}[]
\item[]
    \texttt{$S0$ := ($a$ rightrotate $2$) xor ($a$ rightrotate $13$) xor ($a$ rightrotate $22$)} \pause 
\item[]
	\texttt{$maj$ := ($a$ and $b$) xor ($a$ and $c$) xor ($b$ and $c$)}\pause 
\item[]
	\texttt{$t2$ := $S0$ + $maj$} \pause
\item[]
	\texttt{$S1$ := ($e$ rightrotate $6$) xor ($e$ rightrotate $11$) xor ($e$ rightrotate $25$)} \pause
\item[]
	\texttt{$ch$ := ($e$ and $f$) xor ((not $e$) and $g$)} \pause
\item[]
	\texttt{$t1$ := $h$ + $S1$ + $ch$ + $k[i]$ + $w[i]$} \pause
\item[]
        $h$ := $g$ \\
        $g$ := $f$ \\
        $f$ := $e$ \\
        $e$ := $d$ + $t1$ \\
        $d$ := $c$ \\
        $c$ := $b$ \\
        $b$ := $a$ \\
        $a$ := $t1$ + $t2$
\end{itemize}

\end{itemize}

\end{frame}
%%%%%%%%%%%%%%%%%%%%%%%%%%%%%%%%%%%%%%%%%%%%%%%%%%%%%%
%%%%%%%%%%%%%%%%%%%%%%%%%%%%%%%%%%%%%%%%%%%%%%%%%%%%%%

\begin{frame}{Hauptschleife}
\texttt{$h0$ := $h0$ + $a$ \\
    $h1$ := $h1$ + $b$ \\
    $h2$ := $h2$ + $c$ \\
    $h3$ := $h3$ + $d$ \\
    $h4$ := $h4$ + $e$ \\
    $h5$ := $h5$ + $f$ \\
    $h6$ := $h6$ + $g$ \\
    $h7$ := $h7$ + $h$} \newline \} \newline \} $//Ende$ $der$ $foreach$-$Schleife$
\end{frame}
%%%%%%%%%%%%%%%%%%%%%%%%%%%%%%%%%%%%%%%%%%%%%%%%%%%%%%
%%%%%%%%%%%%%%%%%%%%%%%%%%%%%%%%%%%%%%%%%%%%%%%%%%%%%%

\begin{frame}{Ausgabe}
\texttt{digest = hash = $h0$ append $h1$ append $h2$ append $h3$ append $h4$ append $h5$ append $h6$ append $h7$}
\end{frame}
%%%%%%%%%%%%%%%%%%%%%%%%%%%%%%%%%%%%%%%%%%%%%%%%%%%%%%
%%%%%%%%%%%%%%%%%%%%%%%%%%%%%%%%%%%%%%%%%%%%%%%%%%%%%%
\begin{frame}{Darstellung des Algorithmus}
\note[item]{SHA2 teilt erst in messages,dann in 64 byte chunks, wendet mathematische funktionen auf jeden chunk, und f"ugt diese dann dem hash hinzu.}
\note[item]{der algorithmus tut dies f"ur jeden chunk und f"ugt es dann dem hash wert hinzu.}
\note[item]{am ende bekommt man einen hash wert mit einer genauen l"ange}
\includegraphics[scale=0.3]{sha256.png}\\
\end{frame}
%%%%%%%%%%%%%%%%%%%%%%%%%%%%%%%%%%%%%%%%%%%%%%%%%%%%%%
%%%%%%%%%%%%%%%%%%%%%%%%%%%%%%%%%%%%%%%%%%%%%%%%%%%%%%
\section{\scshape Anwendung}
\subsection{Verwendungszweck}
\begin{frame}{Verwendungszweck}

\begin{itemize}
\item Digitale Zertifikate und Signaturen 
	\pause
\item Passwortverschl"usselung
\begin{itemize}
	\item pam\textunderscore unix: sha2, md5
	\item htpasswd(Apache): sha1, md5
	\item MySQL: sha1
\end{itemize}
\pause
\item Pr"ufsummen bei Downloads
\end{itemize}


\end{frame}
%%%%%%%%%%%%%%%%%%%%%%%%%%%%%%%%%%%%%%%%%%%%%%%%%%%%%%
%%%%%%%%%%%%%%%%%%%%%%%%%%%%%%%%%%%%%%%%%%%%%%%%%%%%%%
\subsection{Schwachstellen/Angriffsvektoren}
\begin{frame}{Schwachstellen/Angriffsvektoren}
\begin{itemize}
\note[item]{bis zum heutigen tage noch keine sicherheitsl"ucken,daher eher schwachstellen als sicherheitsl"ucken}
\note[item]{allgemein wird in starke und schwache hashefunktionen unterschieden}
\item Resistenzen:
\begin{itemize} 
\item Kollisionsresistenz
\item Preimage Resistenz 
\item Second Preimage Resistenz
\end{itemize}
\end{itemize}
\note[item]{wieder auf folie gucken}
\note[item]{sha2 fragil, kleine "anderung im algo  $\to$ grosse auswirkung auf sicherheit}
\note[item]{\tiny{siehe quelle 14}}
\end{frame}
%%%%%%%%%%%%%%%%%%%%%%%%%%%%%%%%%%%%%%%%%%%%%%%%%%%%%%
%%%%%%%%%%%%%%%%%%%%%%%%%%%%%%%%%%%%%%%%%%%%%%%%%%%%%%
\begin{frame}{Kollisionsresistenz}
Wie schwer ist es, zwei verschiedenen Nachrichten mit gleicher Pr"ufsumme zu finden?\\
\includegraphics[scale=0.52]{collision-resistance.png}
\note[item]{kollisionsfreiheit unm"oglich, es muss nur schwer sein welche zu finden: da es unendlich viele eingaben, aber nur endlich viele hashes gibt}
\note[item]{schwache kr: vorgegebene eingabe, gesucht ist eine eingabe, die zum gleichen hash f"uhrt}
\note[item]{schwache kr: gesucht sind zwei beliebige eingaben, die zum gleichen hash f"uhrt}
\end{frame}

%%%%%%%%%%%%%%%%%%%%%%%%%%%%%%%%%%%%%%%%%%%%%%%%%%%%%%
%%%%%%%%%%%%%%%%%%%%%%%%%%%%%%%%%%%%%%%%%%%%%%%%%%%%%%
\begin{frame}{Preimage Resistenz}
Wie schwer ist es, zu einem vorgegebenen Hash-Wert eine Nachricht zu erzeugen, die denselben Hash-Wert ergibt?\\
\includegraphics[scale=0.52]{preimage-resistance.png}
\end{frame}

%%%%%%%%%%%%%%%%%%%%%%%%%%%%%%%%%%%%%%%%%%%%%%%%%%%%%%
%%%%%%%%%%%%%%%%%%%%%%%%%%%%%%%%%%%%%%%%%%%%%%%%%%%%%%
\begin{frame}{Second Preimage Resistenz}
Wie schwer ist es, zu einer vorgegebene Nachricht einen Hash-Wert eine Nachricht zu finden, die denselben Hash-Wert ergeben? \\
\includegraphics[scale=0.52]{2nd-preimage-resistance.png}
\end{frame}

%%%%%%%%%%%%%%%%%%%%%%%%%%%%%%%%%%%%%%%%%%%%%%%%%%%%%%
%%%%%%%%%%%%%%%%%%%%%%%%%%%%%%%%%%%%%%%%%%%%%%%%%%%%%%
\section{\scshape Ausblick}
\subsection{SHA-3}
\begin{frame}{SHA-3}
\begin{itemize}
	\item 2007 rief NIST zu einem Wettbewerb auf
\pause	
	\item 191 Einreichungen, 5 Finalisten
\pause
\item bisher langsamer als SHA2
\end{itemize}
 \note[item]{sha3 finalisten allesamt nicht von attacken gegen prinzipielles \glqq merkle damgard \grqq  verfahren betroffen}
 \note[item]{merkle damgard: \glqq Aus den Nachrichtenbl"ocken wird durch wiederholte Anwendung der Kompressionsfunktion der Hashwert erzeugt \grqq }
\note[item]{bisher langsamer, eventuell unn"otig da sha2 ungebrochen + hoher migrationsaufwand}
 \end{frame}
 
 
%%%%%%%%%%%%%%%%%%%%%%%%%%%%%%%%%%%%%%%%%%%%%%%%%%%%%%
%%%%%%%%%%%%%%%%%%%%%%%%%%%%%%%%%%%%%%%%%%%%%%%%%%%%%%
\begin{frame}
\huge{Fragen?}
\end{frame}

%%%%%%%%%%%%%%%%%%%%%%%%%%%%%%%%%%%%%%%%%%%%%%%%%%%%%%
%%%%%%%%%%%%%%%%%%%%%%%%%%%%%%%%%%%%%%%%%%%%%%%%%%%%%%


\end{document}