%% Einleitung.tex
%% $Id: einleitung.tex 61 2012-05-03 13:58:03Z bless $
%%

\chapter{Einleitung}
\label{ch:Einleitung}
%% ==============================

In der digitalen Gesellschaft und im Zeitalter des Internets wird es immer wichtiger immer mehr Daten so schnell wie möglich zu speichern. Doch die Speicherung ist nicht eine Erfindung der Neuzeit. \textit{Ein Speicher (v. lat.: spicarium Getreidespeicher, aus spica Ähre)[...] ist ein Ort oder eine Einrichtung zum Einlagern von materiellen oder} \textbf{\textit{immateriellen}} \textit{Objekten.} \cite{wiki:Speicher}
\newline
Ein Speichermedium dient also zur kurz- oder langfristigen \glqq \textit{Einlagerung}\grqq{} bzw. Erhaltung von immateriellen Objekten – oder anders ausgedr"uckt Informationen.   


%% ==============================
\section{Zielsetzung der Arbeit}
%% ==============================
\label{ch:Einleitung:sec:Zielsetzung}



%% ==============================
\section{Gliederung der Arbeit}
%% ==============================
\label{ch:Einleitung:sec:Gliederung}

Was enthalten die weiteren Kapitel?

Bla fasel\ldots

%%% Local Variables: 
%%% mode: latex
%%% TeX-master: "thesis"
%%% End: 
