%% Einleitung.tex
%% $Id: einleitung.tex 61 2012-05-03 13:58:03Z bless $
%%

\chapter{Einleitung}
\label{ch:Einleitung}
%% ==============================

In der digitalen Gesellschaft und im Zeitalter des Internets wird es immer wichtiger immer mehr Daten so schnell wie möglich zu speichern.  \textit{Ein Speicher (v. lat.: spicarium Getreidespeicher, aus spica Ähre)[...] ist ein Ort oder eine Einrichtung zum Einlagern von materiellen oder} \textbf{\textit{immateriellen}} \textit{Objekten.} \cite{wiki:Speicher}
\newline
Ein Speichermedium dient also zur kurz- oder langfristigen \glqq \textit{Einlagerung}\grqq{} bzw. Erhaltung von immateriellen Objekten – oder anders ausgedr"uckt Informationen. 
Es stellt sich die Frage: Was sind Informationen? \newline
Im Laufe der Geschichte wurde der Informationsbegriff immer wieder neu definiert. Für die Informatik ist die Beschreibung nach Claude Elwood Shannon\footnote[1]{Claude Elwood Shannon: * 30. April 1916 in Petoskey, Michigan; \textdagger{} 24. Februar in Melford, Massachusetts gilt als Begr"uender der Informationstheorie} relevant. Demnach muss man ein Zeichen als kleinste Informationseinheit und dessen statistische H"aufigkeit in einem Code als Information sehen. Die Information darf nicht mit dem Bedeutungsgehalt verwechselt werden. Eine Information die wenig Sinn ergibt ist einer Information mit großem Sinngehalt gleichwertig.   
    was sind informationen??daten??
    
    unterschied blabla: technische und nichttechnische => doch die speicherung blaaa
    
Doch die Speicherung ist nicht eine Erfindung der Neuzeit. Gespeichert wurde um nicht zu vergessen

%% ==============================
\section{Zielsetzung der Arbeit}
%% ==============================
\label{ch:Einleitung:sec:Zielsetzung}



%% ==============================
\section{Gliederung der Arbeit}
%% ==============================
\label{ch:Einleitung:sec:Gliederung}

Was enthalten die weiteren Kapitel?

Bla fasel\ldots

%%% Local Variables: 
%%% mode: latex
%%% TeX-master: "thesis"
%%% End: 
