%% Einleitung.tex
%% $Id: einleitung.tex 61 2012-05-03 13:58:03Z bless $
%%

\chapter{Einleitung}
\label{ch:Einleitung}
%% ==============================
Im Zeitalter der digitalen Welt und der immer schneller voranschreitenden Entwicklung der Technik, ist es 
notwendig immer größere Datenmenge in immer kürzerer Zeit zu speichern.  
Speichermedien gibt es schon so lange man denken kann und nicht erst seit Beginn der Digitalisierung. 
Speichermedien dienen zur kurz- oder langfristigen  Speicherung von Informationen. Bei dieser 
Datenspeicherung unterscheidet man zwischen der Nichttechnischen Speicherung und der Technischen 
Speicherung der Daten.  
Was sind Informationen und was sind Daten in der Informatik? Diese Frage muss vor Beginn der folgenden
Ausführungen noch geklärt werden.  
Informationen wurde im Laufe der Zeit immer wieder anders definiert, für die Computertechnik wichtig ist die 
Formulierung von Claude Shannon, der Informationen  als statistischen Aspekt betrachtet, um die 
Wahrscheinlichkeit auszuwerten, mit welcher Häufigkeit ein Zeichen auftritt. Es wird gesagt, dass je geringer die 
Auftrittswahrscheinlichkeit eines Zeichens im Code ist, desto höher ist sein Informationsgehalt.  
Daten sind laut DIN ISO/IEC 2382 „Gebilde aus Zeichen oder kontinuierliche Funktionen, die aufgrund 
bekannter oder unterstellter Abmachungen Informationen darstellen, vorrangig zum Zweck der Verarbeitung und 
als Ergebnis.“
1
Einfacher ausgedrückt laut Gumm/Sommer „Als Daten bezeichnen wir, […], die Folgen von Nullen und Einsen, 
die irgendwelche Informationen repräsentieren“
2
. Daten sind also die verschlüsselte Form von Informationen. 
Bei der Nichttechnischen Speicherung werden die Informationen von Hand oder mithilfe eines Trägermaterials 
abgespeichert. Der Vorteil dieser Speichermethode besteht darin, dass man ohne technische Hilfsmittel, die 
Daten sofort wieder lesen kann. Beispiele hierfür sind Papyrusrollen oder auch Bücher. 
Im Rahmen der Technischen Speicherung werden die Informationen mithilfe der Technik abgespeichert und 
können auch nur durch die Technik ausgelesen und für den Menschen nutzbar gemacht werden. 
Erinnern wir uns an die Höhlenmenschen, wie haben Sie ihr Wissen weiter gegeben? Richtig, in Form von 
Höhlenmalerei. Die alten Ägypter nutzten Papyrusrollen. Der deutsche Goldschmied Johannes Guttenberg 
entwickelte dann im Jahr 1440 die Buchdruckpresse,  damit wurden alle Informationen, dann in Büchern 
„abgespeichert“.  
So ging die Entwicklung immer weiter, von Büchern,  über Lochkarten, Magnetbändern, hin zu den heute 
bekanntesten Speichermedien wie Festplatten, Arbeitsspeicher, CDs, DVDs, Blu-Rays und USB-Sticks.  
Im Jahr 1950 schätzte der Physiker Richard Feynman die Zahl der Buchtitel der Welt auf ca. 24 Millionen und 
formulierte dabei die Vision, dass es irgendwann möglich wäre, den Inhalt all dieser Bücher auf ein einziges 
Staubkorn zu speichern. Laut seiner Vision ist das  Staubkorn noch mit bloßem Auge sichtbar. Diese Vision 
klingt unglaublich, aber ist sie vielleicht doch realisierbar? 
Die folgenden Seiten sollen eine kleine Zusammenstellung der Speichermedien geben und auch ihre Nutzung, 
sowie Vor- und Nachteile aufzeigen.  
Diese Arbeit ist eine fachliche Zusammenstellung der bereits bekannten Informationen. Am Ende steht ein 
Ausblick darauf wie sich die Speichermedien weiterentwickeln könnten.  
Bla fasel\ldots



%% ==============================
\section{Zielsetzung der Arbeit}
%% ==============================
\label{ch:Einleitung:sec:Zielsetzung}

Was ist die Aufgabe der Arbeit?

Bla fasel\ldots

%% ==============================
\section{Gliederung der Arbeit}
%% ==============================
\label{ch:Einleitung:sec:Gliederung}

Was enthalten die weiteren Kapitel?

Bla fasel\ldots

%%% Local Variables: 
%%% mode: latex
%%% TeX-master: "thesis"
%%% End: 
