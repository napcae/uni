%% technisch.tex
%% $Id: technisch.tex 61 2012-05-03 13:58:03Z bless $
%%

\chapter{Technische Speicherung}
\label{ch:Technische Speicherung}
%% ==============================
Die Grundlagen müssen soweit beschrieben
werden, dass ein Leser das Problem und
die Problemlösung  versteht.Um nicht zuviel 
zu beschreiben, kann man das auch erst gegen 
Ende der Arbeit schreiben.

Bla fasel\ldots

    %% ==============================
    \section{Fotografische Speicherung}
    %% ==============================
    \label{ch:Technisch:sec:Fotografische Speicherung}
    %% ==============================
        
Die Informationen beziehungsweise Daten werden in sogenannten Lichtbildern, in der Gemeinsprache auch Bild, Film oder Foto genannt in einem chemischen Prozess gespeichert. Diese chemooptische Speicherform l"asst sich einfach mit einem Vergr"o"serungsglas betrachten. Zurzeit ist diese Form von Speicherung die sicherste und langlebigste Archivierungsmethode. Ein so hergestellter Mikrofilm h"alt bis zu 400 Jahre.
                
    %% ==============================
    \section{Mechanische Speicherung}
    %% ==============================
    \label{ch:Technisch:sec:Mechanische Speicherung}
    %% ==============================
    
    Bei der mechanischen Speicherung handelt es sich um Speichermedien die im Fertigungsprozess physische bearbeitet werden. Die Daten werden durch Erh"ohungen oder Vertiefungen auf das Tr"agermaterial gepresst und sind somit Read-Only Speicher, d.h. Daten die einmal auf diese Medien gebracht wurden, kann man nicht mehr "andern. Grunds"atzlich unterscheid man bei der mechanischen Speicherung in analoge und digital. Nachteil ist abnutzung durch mechanische reibung. Das klassische Beispiel ist hier die Vinyl LP\footnote[4]{engl. LP = Longplayer = Langspielplatte} oder CD. 
        \subsection{Lochkarte}
        %% ==============================
        \label{ch:Technisch:sec:Mechanische Speicherung:sub:Lochkarte}
        %% ==============================
            
            Bla fasel\ldots
    
    
    %% ==============================
    \section{Magnetische Speicherung}
    %% ==============================
    \label{ch:Technisch:sec:Magnetische Speicherung}
    %% ==============================
        \subsection{Magnetb"ander}
        %% ==============================
        \label{ch:Technisch:sec:Magnetische Speicherung:sub:Magnetb"ander}
        %% ==============================
        
            Bla fasel\ldots
            
        \subsection{Kernspeicher}
        %% ==============================
        \label{ch:Technisch:sec:Magnetische Speicherung:sub:Kernspeicher}
        %% ==============================
        
            Bla fasel\ldots
            
        \subsection{Festplatte}
        %% ==============================
        \label{ch:Technisch:sec:Magnetische Speicherung:sub:Festplatte}
        %% ==============================
        
            Bla fasel\ldots
    
    %% ==============================
    \section{Elektronische Speicherung}
    %% ==============================
    \label{ch:Technisch:sec:Elektronische Speicherung}
    %% ==============================
        \subsection{DRAM und SRAM}
        %% ==============================
        \label{ch:Technisch:sec:Elektronische Speicherung:sub:DRAM und SRAM}
        %% ==============================
        
            Bla fasel\ldots
        
        \subsection{ROM}
        %% ==============================
        \label{ch:Technisch:sec:Elektronische Speicherung:sub:ROM}
        %% ==============================
        
            Bla fasel\ldots
            
        \subsection{Flash-EEPROM}
        %% ==============================
        \label{ch:Technisch:sec:Elektronische Speicherung:sub:Flash-EEPROM}
        %% ==============================
        
            Bla fasel\ldots
    
    %% ==============================
    \section{Optische Speicherung}
    %% ==============================
    \label{ch:Technisch:sec:Optische Speicherung}
    %% ==============================
        \subsection{CD}
        %% ==============================
        \label{ch:Technisch:sec:Optische Speicherung:sub:CD}
        %% ==============================
        
            Bla fasel\ldots
        
        \subsection{DVD}
        %% ==============================
        \label{ch:Technisch:sec:Optische Speicherung:sub:DVD}
        %% ==============================
        
            Bla fasel\ldots
        
        \subsection{Blu-Ray}
        %% ==============================
        \label{ch:Technisch:sec:Optische Speicherung:sub:Blu-Ray}
        %% ==============================
        
            Bla fasel\ldots

%%% Local Variables: 
%%% mode: latex
%%% TeX-master: "thesis"
%%% End: 
