%% eval.tex
%% $Id: eval.tex 61 2012-05-03 13:58:03Z bless $
% !TEX root = thesis.tex

\chapter{Evaluierung}
\label{ch:Evaluierung}
%% ==============================
Hier kommt der Nachweis, dass das in Kapitel~\ref{ch:Entwurf}
entworfene Konzept auch funktioniert. Leistungsmessungen einer
Implementierung werden auch immer gerne gesehen.

Bla fasel\ldots

%% ==============================
\section{Abschnitt 1}
%% ==============================
\label{ch:Evaluierung:sec:Abschnitt1}

Bla fasel\ldots

%% ==============================
\section{Abschnitt 2}
%% ==============================
\label{ch:Evaluierung:sec:Abschnitt2}

Bla fasel\ldots

%% ==============================
\section{Zusammenfassung}
%% ==============================
\label{ch:Evaluierung:sec:zusammenfassung}

Am Ende sollten ggf. die wichtigsten Ergebnisse nochmal in \emph{einem}
kurzen Absatz zusammengefasst werden.

%%% Local Variables: 
%%% mode: latex
%%% TeX-master: "thesis"
%%% End: 
